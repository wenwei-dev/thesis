\subsubsection{对话系统}{Dialogue Systems}

图灵曾在其1950年的论文中阐述了他对终极NLP应用的预期:对话系统能够以自然的方式与人类使用者进行谈话。粗糙的对话系统(如:Apple的SIRI),在今天已成为我们日常生活的一部分,但真正具有人类思维能力的对话系统仍然没有实现其研究目标。

图\ref{fig:dialogue}描述了对话系统的通用结构\cite{Arora2013}。除了NL生成和理解,以及从各种资源中获取相关知识,其主要模块是对话管理器。这个模块决定在谈话中的每个节点说什么。

\begin{figure}[htb]
\centering
\includegraphics[width=12cm]{figures/dialogue_system.png}
\caption{ A generic dialogue-system architecture }
\label{fig:dialogue}
\end{figure}

总之,对话管理器全方位管理对话。它采用用户文本的语义表示、确定文本是否契合上下文,并构建系统响应的语义表示。一般来说,在它的职责中,以下几个是必要的:

\begin{itemize}
\item 保存对话历史
\item 采用一定的对话策略
\item 处理格式错误和无法识别的文本
\item 检索文件或数据库中的内容
\item 确保为使用者提供最好的响应
\item 管理启动和系统响应
\item 处理语言学问题
\item 言语分析
\item 以任何可用的、相关的非语言信息来构建语言结构。
\end{itemize}

目前大多数对话系统都利用一小部分固定规则来处理对话管理(这些规则代表人工制定的言语策略)。近年的一些系统还提供另一种方法:利用概率模型(如:POMDP)来控制对话,但这些系统还不具备有价值的实用性\cite{Young2006} \cite{Williams2010}。

\begin{figure}[htb]
\centering
\includegraphics[width=12cm]{figures/pomdp.png}
\caption{ A POMDP-based reinforcement-learning-driven dialogue-system architecture }
\label{fig:dialogue}
\end{figure}



\subsubsection{问答系统}{Question Answering}

就连接自然语言理解、生成和知识表示而言,也许最简单、最有意义的模式就是“问答系统”。问答(QA)系统是一种特殊形式的对话系统,它让使用者以自然语言提问,然后以自然语言响应。


许多现代QA系统基本上都是以文档为驱动的信息检索系统,请见图\ref{fig:qa}中的示例\cite{Hirschman2001}。这种系统(其中的问题被直接提交至一个大规模文本语料库)的表现通常优于20世纪70年代的早期QA系统(依赖人工编码的知识库)。

\begin{figure}[htb]
\centering
\includegraphics[width=12cm]{figures/qa.png}
\caption{典型的文件驱动问答系统的架构}
\label{fig:qa}
\end{figure}

这种以文件为驱动的QA系统通常包括一个问题分类模块。这个模块确定问题和回答的类型。分析问题之后,系统一般会运用几个模块。文本的数量逐步减少,这些模块则越来越多地应用于复杂的NLP技术。因此,文件检索模块可以使用搜索引擎来识别那些可能含有“答案”的文件,或文档中的段落。随后,一个过滤器预先挑选出包含同类字符的小文本碎片,用作预期回答。例如:如果问题是“谁发明了青霉素?”,过滤器会反馈包含人名的文本。最后,一个“回答抽取模块”会在文本中进一步寻找线索,以确定候选的答案是否能正确地回答这个问题。

另一方面,如果问答系统会根据对文本知识的深层理解来做出响应,那么“快速处理文本来响应问题”就不是一个非常可行的策略。更准确地说,用于为QA系统提供知识的文件必须预先由一个NL理解系统彻底地“读”和理解。后面的那个方法是我们要在本文的第\ref{chap:dialogue}章中探讨的。我们开发了一个简单的QA系统,它是一个整体互动对话体系的一部分。

在2010年,IBM的问答系统Watson以极大的优势击败了另两个Jeopardy冠军奖的获得者:Brad Rutter 和 Ken Jennings。Watson综合利用了信息检索法和基于推理的先进方法\cite{Ferrucci2011},可以说是迄今为止最先进的问答系统。

\subsubsection{问答系统的关键问题}{Key Issues Regarding Question Answering}

问答系统是一个复杂的探索课题,它涉及众多问题。举例来说,有证据表明:某种问题要难于其它问题。询问“为什么”和“怎样”的问题往往比询问“是什么”和“在哪里”的问题更难回答,这是因为它们要求对因果关系,或instrumental关系的理解。这些关系通常由分句或独立的句子来表达\cite{Hirschman1999}。

在2002年,一组研究人员绘制了一幅关于问答系统的研究路线图\cite{Burger2002}。他们当时提出的问题在今天仍然与我们息息相关。以下是他们发现的问题:

\begin{itemize}
\item 问题类别:不同类型的问题(例如“列支敦士登的首都是哪里?” VS“为什么会形成彩虹?”VS“玛丽莲·梦露和加里·格兰特出演过同一部电影吗?”)要求使用不同的策略来发现答案。
\item 问题处理:相同的信息可能是用不同的方式 来表达的,有些是疑问句(“莱索托国的国王是谁?”),有些是祈使句(“告诉我莱索托国王的名字。”)。因此,梳理出有效信息需要花些功夫。
\item 上下文和问答:上下文可能用于厘清某个问题、解决歧义,或者追踪一系列问题的调查。(例如“为什么Joe Biden2010年访问了伊拉克?”。这个问题也可能这样问:为什么副总统Biden去访问,而不是Obama总统;为什么他去的是伊拉克,而不是阿富汗或其他国家;为什么他是2010年去的,而不是在那之前或之后;或者Biden希望在那次访问中取得什么成果等。)
\item 回答公式:QA系统生成的结果以尽可能自然的方式呈现。
\item 实时问答:不论问题有多复杂,“快速回答”在实际应用中都非常重要。
\item 互动问答:经常出现的一种情况是:QA系统没有很好地获取信息需求,因为问题处理部分可能没有成功地对需要抽取的问题或信息进行正确分类,而且生成答案也不容易检索。在这种情况下,提问的人可能不仅想重新表达问题,还想与系统进行对话。此外,系统也可以利用之前回答过的问题。
\item 先进的QA推理:更多复杂的问题等待着书面文本或结构数据库以外的回答。为了完善QA系统,在其中添加这些功能,以下几项工作是必要的:整合推理模块、建立多种知识库、对通用知识,常识推理机制,以及不同领域的特定知识进行编码。我们的研究正是要推动这方面的发展。
\item QA用户归档:用户归纳是用于获取提问者的数据,包括上下文数据、兴趣、提问者常用的推理方案、系统和使用者之间不同对话的共同点等。这种归档可以为QA系统的表现提供有价值的指引。
\end{itemize}

另一个与QA有关的问题是:如何判断某个“回答”的质量。以下是几个常用标准:

\begin{itemize}
\item 相关性:“回答”应该是对问题的响应。
\item 正确性:“回答”应该符合事实。
\item 简洁性:“回答”不应该包括无关信息。
\item 完整性:“回答”应该完整,例如:不完整的回答不应该得满分。
\item 连贯性:“回答”应该是连贯的,这样才能方便提问者阅读。
\item 正当理由:“回答”中应该带有足够的上下文,以便提问者了解为什么选择了这个答案。
\end{itemize}

到目前为止,大多数研究都关注了“相关性”。