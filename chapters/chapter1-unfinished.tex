

从人工智能发展的早期开始,实现人机之间的自然语言交互一直是该领域的研究焦点,1950年,Alan Turing提出了一个如何判定机器是否具有智能的标准“图灵测试”。至此,对话系统更是成为判定机器是否智能的理想模型。然而,对话系统一直是自然语言处理和人工智能领域的研究难题,它不仅涉及到对自然语言的处理,自然的对话系统还要求机器具备一定的认知能力,使其能按照人的响应方式与人类进行交互。本文从目前对话系统研究中亟待解决的认知技术角度出发,对此展开深入研究。

本章简要介绍智能对话系统的研究背景和意义、以及相关的研究现状和存在的问题,最后列出本文的主要研究内容和组织结构。

%%%%%%%%%%%%%%%%%%%%%%%%%%%%%%%%%%%%%%%%%%%%%%%%%%%%%%%%%%%%%%%
\section{研究背景和意义}{Background}
%%%%%%%%%%%%%%%%%%%%%%%%%%%%%%%%%%%%%%%%%%%%%%%%%%%%%%%%%%%%%%%

       与机器进行智能对话一直是人们梦寐以求的梦想。随着人工智能技术的不断发展,对话系统逐渐成为人工智能在商业领域的焦点,各大公司也都纷纷高调发布对话系统产品,如在Jeopardy节目中夺冠的IBM Waston,苹果的Siri,微软的Cortana和小冰,小i和图灵等等。尽管近年来,对话系统相关的计算语言学和人工智能领域的很多技术都取得了巨大的进展,如语音识别和合成、机器翻译、信息检索、浅层语义分析甚至情感分析等,但对话系统的认知对话管理和语义语用等深层理解领域仍没有根本性的突破。即使是目前表现最好的对话系统,也依然是基于表层的语言分析和简单的启发性知识来实现。然而如果对话系统无法作为一个认知主体,以拟人的方式去感知和交互,就很难激发人类与其长时间交流的兴趣。
   
       对话系统涉及的领域很广,本文重点考虑认知型的对话系统,也就是强调机器作为一个认知主体,去理解和适应人的交流方式,从而使用户能用与人交流的自然方式去和机器交流。本文中提到的对话系统中的相关认知技术主要包括知识的深层表示、不确定性逻辑推理、语言与涉身知识的融合以及动机驱动的对话控制。对这些领域的研究显然具有重要的研究价值。这些领域中任一组块取得进展,不仅仅对于智能对话系统,而且对于其他领域,如语义理解、情感建模、自动化学习系统等等,都会有相当大的辅助和促进作用。另一方面,对话系统中的认知技术得到改进,无疑能大幅度提高对话系统的实用价值。能帮助机器更好更准确地结合语境理解和适应用户的交流方式,从而更准确更人性化地帮助人类完成一系列的辅助工作,拟人的交流方式也会使人们觉得更舒心和信赖,这也将为人们的生活带来诸多便利。

         尽管拥有巨大的研究价值和市场价值,研究者也发现构建一个能达到人类水平的对话系统是一项十分艰巨的任务。本文的研究着重于对话系统中的相关认知技术,提出了一个动机驱动的对话控制模型,即结合了心理学上的动机驱动模型,言语行为理论,以及基于概率逻辑网络的超图形式的知识表示,赋予对话系统一定的认知能力,从而使机器能作为一个认知主体,以更自然的方式与用户进行交互。


%%%%%%%%%%%%%%%%%%%%%%%%%%%%%%%%%%%%%%%%
DSC树部分


鉴于将自然语言话语转换成上述两种表示形式非常困难,Liang等人\cite{Liang2013}试图将该任务简化成一个允许进行高效推理的约束满足的问题。为此,他们提出了一种新的知识表示方式:基于依存的组合语义树(Dependency based  Compositional Semantics Tree,以下简称DCS树),并通过预先设定的触发词和谓词的约束来高效地将自然语言转换成DCS树的表示形式,如单词city将会触发相应的谓词city。DCS树通过树的形式来构建逻辑表达式,它与依存句法树平行,为自然语言的语义解析提供了很多便捷。尽管该方法的提出只针对小规模数据集合上和预先设定的触发词表,但不难通过语义角色标注和浅层语义分析抽取出语句中的谓词等自动方法来代替这些人工编写的词汇表,从而扩大该方法的应用规模。此外,该表示方法还能结合大规模的事实数据库,如Freebase\cite{Bollacker2008}或者Google的知识图表,可以通过简单的语义角色标注和基于对先前观察的语句来填充语义空槽,从而实现对现有问答系统的扩展。


尽管DCS树的表达方式在有限领域和数据集合上能达到较好的效果,但从其研究\cite{Liang2011}来看,该表示方法的表达能力仍然是个悬而未决的问题,目前也没有任何相关研究表明其表达能力与一阶谓词逻辑的表示方法相当。从另一个角度来看,DCS树的表达方式受数据库查询语言的启发\cite{Yarin2013},通过数据库查询语言来表示自然语言语句的语义为知识表示提供了一个全新的思路,并不仅因为它将数据库查询语言的成熟研究也应用到语义表示和学习方面,而且它也实现了在约束满足条件下有效进行逻辑推理和蕴涵\cite{Giordani2010a}, \cite{Giordani2010b}, \cite{Giordani2009}。然而即使数据库查询语言相当成熟,但其表达能力仍然是非常有限的,因为这些语言除了无法表示聚合函数和相关的算术运算,还无法表达递归查询\cite{Libkin2003},但递归结构在自然语言中显然是相当常见的,这在一定程度上间接说明了DCS树的表达能力不一定足够用于实用的自然语言处理应用领域。


%%%%%%%%%%%%%%%%%%%%%%%%%%%%%%%%%%%%%%%%%%

语义网表示部分

语义网络的知识表示是人工智能和自然语言处理等领域的又一种经典的知识表示方式,它主要用于表示陈述性知识。语义网络在形式上是一个加权有向图,其中有向图的结点表示某个概念、对象、事件或状态等,边(也称为弧)则表示结点之间的语义关系。

语义网络中的语义关系通常包括实例关系(ISA,用于表示实体结点和类结点之间的关系,如“史努比”和“狗”)、泛化关系(AKO,用于表示类结点与抽象层次更高的类结点,如“鲸鱼”和“哺乳动物”)、部分整体关系(Part-of,用于表示结点与其组成成分之间的关系)以及属性值关系(用于表示个体的各属性和对应的属性值之间的关系,通常使用语义网络中的边来表示属性,用边所指向的结点来表示该属性的值)。除此之外,语义网络还能表示以谓词或关系为中心组织语义关系,还可表示不同子网络之间的逻辑关系,如析取、合取、否定和蕴涵等。

语义网络的知识表示体系所采用的推理机制主要有匹配和继承,继承一般通过匹配和搜索来实现。在问题求解时,可以根据问题的描述来构造一个语义网络片段,并在语义网络知识库中查找可以与该语义网络片段匹配的子网络,然后根据查找到的匹配来回答问题。

目前被广泛使用的基于语义网络的知识表示库有ConceptNet\cite{Liu2004}和谷歌的知识图谱等。语义网络非常符合人类的思维习惯,其表达方式自然、简洁,易于理解。表达能力方面,语义网络不仅可以表示事物的属性状态、行为动作、目标功能等,而且还能表示事物之间的关联,为人工智能和自然语言处理等相关系统提供了便捷的知识推理平台,但是语义网络也面临着诸多挑战,首先,语义网络的形式过于简单,很难表达相对复杂的关系类型,复杂关系的引入很容易增加语义网络的复杂度,从而使知识的储存和检索过程变得相当复杂,甚至难以实现。其次,语义网络中的语义关系查询和推理往往需要计算复杂度非常高的图算法,而且基于图的算法一般可扩展性差,这就直接导致,在知识库达到一定规模后,将面临着计算效率问题的挑战。最后,该表示方法还面临着严重的数据稀疏问题的挑战,对那些在语义网络中对外连接较少的实体,意味着涉及到它们的路径很少,这导致图算法对其相关路径查询和逻辑推理带来很大的难度。


%%%%%%%%%%%%%%%%%%%%%%%%%%%%%%%%%%%%%%%%%%

传统逻辑表示方法和概率逻辑表示的衔接部分

上述的几种表示方法都可以看成是刚性表示方法,因为它们都无法在逻辑推理和蕴涵过程中处理非确定性。然而真实世界中大部分知识都是非确定性的,近年来也有将概率引入到这些传统的知识表示方法中,如\cite{Poon2009} 中将依存树转换成准概率逻辑形式,并使用了马尔科夫逻辑网络 \cite{Domingos2007}进行语义解析,这样一来,该方法既能进行逻辑推理和蕴涵,同时处理其中的不确定性。然而,该表示仅限于表达一些基本形式的语句,\cite{Titov2011}将上述工作中的表示继续简化,在马尔科夫逻辑网络中使用了Pitman-Yor随机过程来对谓词和变量之间的概率依存关系进行建模,使其更适用于简单的角色标注的形式。这些仅仅是在传统逻辑表示上引入了概率随机过程等,下面我们将简单介绍一些基于概率逻辑的知识表示方法。



%%%%%%%%%%%%%%%%%%%%%%%%%%%%%%%%%%%%%%%%%%

基于概率逻辑的知识表示法部分

      随着大数据技术的不断发展,不确定性的知识表示和管理在人工智能各领域越来越被重视起来,如上所述,基于概率逻辑的知识表示使得知识表示库能以认知概率形式来储存世界知识中的不确定性。因此,近年来,基于概率逻辑形式的知识表示方法不断涌现并广泛应用于各类学习系统中,如马尔科夫逻辑网(MLN),归纳逻辑编程(ILP),概率逻辑编程(PLP),贝叶斯逻辑编程(BLP),随机逻辑编程(SLP),随机关系模型(SRM),概率关系模型(PRM)等等。篇幅关系,我们这里只简单介绍和本文所使用的知识表示方法相似的几个典型的基于概率逻辑的表示方法。

      贝叶斯逻辑编程 (Bayesian Logic Programs,以下称BLP)\cite{Kersting2007}采用贝叶斯网络来表示知识,其中节点表示命题子句,它是众多“概率Prolog”中的一种。BLP可以用于表示任意带概率的Prolog形式的关系,并能通过逻辑推理来传播这些概率。BLP中维持了概率层次上的“直接因果推理”和逻辑层次上“间接推理”的同构关系,使得该结构中的概率和逻辑推理能达到一致,而无需借助棘手的混合模型。Puech\cite{Puech2003}对BLP的表达能力进行深入研究,并将其与SLP(Stochastic Logic Programs)的表达能力进行对比,其结果显示SLP能将同样的知识表示成BLP的一个子类。但BLP中的推理机制(即通过结构化的EM算法来查找BLP结构)却比SLP要简单的多,而SLP中的推理机制是被公认为非常困难的\cite{DaRaett2003}。然而,由于BLP依赖于Prolog,也继承了Prolog的不足之处,如依赖于特定的逻辑表达式归一化、缺少可扩展的推理机制等。
     
   马尔科夫逻辑网(Markov Logic Network,以下称MLN)最初由Domingos等人提出\cite{Domingos2007},使用


MLN


 Leading figures in the field have adopted this approach
(Huynh and Mooney, 2008; Mihalkova, Huynh, and Mooney, 2007; Mihalkova and Mooney,
2007), however due to the difficulties in inference this approach is losing its popularity
among its advocates (Beltagy, Chau, Boleda, Garrette, Erk, and Mooney,
2013).

Markov Logic Networks (XX) is a novel formalism for inferring the
probability distribution associated with a set of entities related by
propositional logic relationships.   Some of the relationships must be
given empirically derived probabilities, and then the network derives
probabilities for the other relationships, via (in one common
implementation) a Gibbs sampling based algorithm.   The effectiveness
of the algorithm depends on the connectivity pattern of the graph and
the complexity of the distribution involved.  For many problems in NLP
it has proven effective, e.g. for entity extraction and semantic
relationship extraction.   The formalism underlying the algorithm
assumes a single probability distribution underlying the entire
network, which is different philosophically and pragmatically from
approaches like NARS and PLN that avoid comparable assumptions of
global consistency.


     与传统的知识表示方法相比,上述的知识表示方法的有点在于能在逻辑中表示认知的不确定性,然而,这些这些表示结构都是离散的,无法用来推理句子之间的相似度。因此\cite{Brocheler2012}提出了PLP(Probabilistic Similarity Logic )来引入原子之间的距离衡量机制,但目前只能用于简单的不确定性逻辑推理中。本文的认知对话系统中所使用的概率逻辑网(Probabilistic Logic Network,以下称PLN)也有类似的地方,且能够将其他非语言渠道(如机器视觉感知)获得的相似度通过概率推理整合得到句子之间的相似度,并整合到其他的逻辑关系(如继承关系)中。

      Pei的NARS\cite{Wang2006}引擎运用了传统逻辑推理法,并引入了独特的数学理论来管理传统逻辑关系中的不确定性。但NARS是构建在经验的基础上,而不是基于模型轮语义。 在许多方面,我们所使用的PLN逻辑形式化体系与NARS有类似的地方,但也存在巨大差异。PLN在一套独特的数学理论基础上同时采用了传统逻辑和谓词逻辑,并且根据概率论和模糊数学理论来推导出不确定性真值的公式,而NARS的不确定性推理体系则基于原始的非逻辑形式。

     图\ref{fig:nars}展示了基本演绎推理、归纳推理和外展推理公式。这些公式是PLN和NARS共有的。在每个关系式右边的$<s,c>$表示“每个关系的强度和置信度”。PLN和NARS使用不同的公式,从那些前提中推导(优势、信息)结论的真值。

\begin{figure}[htb]
\centering
\includegraphics[width=12cm]{figures/nars.png}
\caption{ NARS/PLN 传统逻辑中演绎推理、归纳推理和回溯推理的形式 }
\label{fig:nars}
\end{figure}
 

%%%%%%%%%%%%%%%%%%%%%%%%%%%%%%%%%%%%%%%%%%
组合分布式表示部分

Distributional methods have been in use in lexical semantics for more than 40
years (Jurafsky and Martin, 2008). They make use of a continuous representation for the words; For each word its neighbouring words are collected resulting
in long binary vectors, often having their dimensionality reduced. 
To capture more complex behaviour, we have to turn to a compositional distributional
representation.

Coecke, Sadrzadeh, and Clark (2010) have recently suggested a novel representation
relying on a combination of the traditional compositional representation
and the continuous distributional one. The representation uses tensor products3
to carry information from word
level to a higher level, and then uses linear functions on these to project into
lower dimensional sentence spaces.

Grefenstette (2013) has shown that this representation can be used to capture
uncertainty as well. He explains how the use of linear projections can be
identified with non-quantified first order logic. However he also proves a negative
result showing that quantified first order logic cannot be obtained when
using only linear projections. He then introduces non-linear mappings to obtain
quantified first order logic.

This representation has been gaining support in the field of semantic modelling
(Van de Cruys, Poibeau, and Korhonen, 2013; Hermann, Grefenstette, and Blunsom,
2013). It offers unambiguous representation, it is expressive, it might allow the
inference of sentiment and role labels (as we will see below), captures uncertainty
and the truth of a sentence, and models the distance between sentences
as points in a sentence space thus captures sentence similarity. Current research
is looking into logical inference and the extraction of the meaning of an
utterance into the sentence space in a manageable way.

CCG
%%%%%%%%%%%%%%%%%%%%%%%%%%%%%%%%%%%%%%%%%%

Probabilistic Representation \cite{Poon2009}

Markov Logic\cite{Domingos2007}

efficient logical inference \& entailment in the form of contraint satisfaction\cite{Giordani2010a},\cite{Giordani2010b}, \cite{Giordani2009}

\cite{Puech2003}have studied
the expressive power of these representations and compared them to the expressive
power of SLPs (which define probabilities over derivations) – showing
that SLPs can encode the same knowledge as a sub-class of BLPs. 

PLP, BLP\cite{Kersting2007}


Probabilistic Similarity Logic\cite{Brocheler2012}

Matthias Brocheler, Lilyana Mihalkova, and Lise Getoor. Probabilistic similarity
logic. arXiv preprint arXiv:1203.3469, 2012.

Probablistic Logic Representation has two contrasting interpretations\cite{DaRaett2003}

Luc De Raedt and Kristian Kersting. Probabilistic logic learning. ACM
SIGKDD Explorations Newsletter, 5(1):31–48, 2003. 

%%%%%%%%%%%%%%%%%%%%%%%%%%%%%%%%%%%%%%%%%%



%%%%%%%%%%%%%%%%%%%%%%%%%%%%%%%%%%%%%%%%%%%%%%%%%%%%%%%%%%%%%%%
\section{存在的问题}{Omissions of Present Research}
%%%%%%%%%%%%%%%%%%%%%%%%%%%%%%%%%%%%%%%%%%%%%%%%%%%%%%%%%%%%%%%

对话系统的研究已有几十年的历史,尽管有不少学者,这些对话系统能够满足一些娱乐性的需求和非常限定场合下的信息查询用途,但仍有许多亟待改进的地方,主要包括:

(1)知识深层表示问题

        认知对话系统首先需要一个能存储丰富语义信息且能灵活操作的深层语义表示体系来支撑,这样一来,对话系统就能从自然语言话语中捕获浅层的句法信息之外的深层语义信息,用于指导对话系统的认知过程。 然而知识表示体系的研究面临着表达能力和计算效率两大挑战,目前很多知识表示体系不得不为了使计算效率达到可用范围而牺牲其表达能力,也就是说,只能表达小规模的知识,一旦知识库达到一定规模,往往出现很多不可操控的问题。

(2)不确定性逻辑推理问题

       不少对话系统中不考虑非精确条件下的语言理解问题,以及在不确定的条件下对话系统如何推断和决策问题。我们认为,具有认知功能的对话过程不仅需要具备一定的逻辑推理功能,即通过演绎、归纳和回溯等推理过程实现的推理,还需要具备不确定推理功能,即涉及概率和模糊逻辑等不确定性的推理,从而使得对话系统能允许非精确输入,并实现在非精确条件下的有效理解,提高对话交互的自然度。

(3)语言和涉身知识的融合问题

       目前的对话系统基本上只考虑语言层面上的交互,但是,自然的对话交流都需要考虑特定的语境和语用等方面的信息,语言的交互也不仅仅是语言层面上的交互,还需要理解语言背后涉及到的涉身交互,也就是说,每一句话语应该被理解成一个“言语行为”,即包含一些独特的语言属性,还包含一些涉及到言语行为和其他类型行为的语用属性,也就是通常所说“言外之意”。

(4)动机驱动的对话控制:

     大多数对话系统都是作为一个执行主体来完成用户指派的对话任务,然而要实现自然的对话交互,还要求机器能作为一个认知主体,能去适应人的交流方式。我们认为,认知对话系统应该根据智能体本身的动机来做决策,并且能在交互的过程中根据人的方式来调整系统动机,以实现更拟人更逼真的对话。也就是说,对话系统在决定什么时候说什么时,必须有一个动机驱动的行为选择模型来指导,而不是基于简单的浅层语言提示来进行回应。


%%%%%%%%%%%%%%%%%%%%%%%%%%%%%%%%%%%%%%%%%%%%%%%%%%%%%%%%%%%%%%%
\section{研究目标和内容}{Goals and Contributions}
%%%%%%%%%%%%%%%%%%%%%%%%%%%%%%%%%%%%%%%%%%%%%%%%%%%%%%%%%%%%%%%
      构建一个能达到人类水平的对话系统是一项十分艰巨的任务,实现这个长远目标远超过本文所能涵盖的范围。本文针对当前相关研究存在的问题,提出了本课题的研究目标:从对话系统的认知需求着手,深入研究知识表示的深层模型以及在该表示上的不确定概率推理,设计并实现自然语言理解和自然语言生成的流程,实现架构从语言到逻辑之间的桥梁。采取动机驱动的情感计算模型和言语行为理论的思想指导,提出一个能使对话系统具有一定认知能力的动机驱动的对话控制模型;并在该模型上实现一个以信息查询为动机的问答系统。

       本文的主要研究内容如下:
(1)研究对话系统的认知需求,借鉴心理学动机模型、言语行为理论以及概率逻辑推理理论等,提出一个动机驱动的对话控制模型。
(2)研究知识的深层表示问题,使其不但能有效储存丰富的语义语境信息,还能被灵活操作。
(3)研究基于深层知识表示的自然语言理解机制,设计并实现一个自然语言理解框架,使其能将自然语言语句转换成(2)中使用的基于超图的抽象的逻辑语义表示形式。
(4)研究基于深层知识表示的自然语言生成机制,设计并实现一个自然语言生成框架,使其能将基于(2)的深层表示的抽象逻辑语义形式转换成自然语言语句。
(5)研究不确定性逻辑推理,将概率逻辑网络应用在深层知识表示结构中,设计和实现一个能在自然语言语句上执行的常识推理系统。
(6)借助上述(3)中的自然语言理解框架,将简单的英文维基百科进行自然语言理解处理并表示成基于超图的深层语义表示形式,并在(1)中的对话控制模型上实现一个以信息查询为动机的问答系统。

%%%%%%%%%%%%%%%%%%%%%%%%%%%%%%%%%%%%%%%%%%%%%%%%%%%%%%%%%%%%%%%
\section{论文的组织结构}{Outline}
%%%%%%%%%%%%%%%%%%%%%%%%%%%%%%%%%%%%%%%%%%%%%%%%%%%%%%%%%%%%%%%

本文按照如下方式组织:

第一章,给出了认知对话系统的研究背景和意义,总结了国内外相关领域的研究现状,并指出了现有研究的局限和存在的问题。最后,给出了本文的研究目标,并介绍了本文的主要研究内容。

第二章,深入探讨了认知对话系统中所使用的深层知识表示体系,并从范畴论角度出发,阐述了能将自然语言映射到该表示体系的理论基础,最后,我们还介绍了能用在该表示体系上不确定性逻辑推理机制,即概率逻辑网。

第三章,主要介绍了动机驱动的情感计算模型Psi,以及该模型中所采用的情感和人性等机制,为下一章节的动机驱动的对话控制模型提供一定的背景知识和理论基础。

第四章,本章首先讨论了如何对心理学上动机驱动的情感计算模型Psi进行改进,使其能用在本文的认知对话控制机制中,然后介绍了认知对话系统的相关目标和动机,提出了引入逻辑推理的动机驱动的行为控制策略,并提出了使用动机驱动来进行对话语篇管理;结合言语行为理论,针对不同的言语行为类型提出了不同的言语行为规划器;最后,结合上述研究成果,提出了动机驱动的对话控制模型。

第五章,介绍本文设计实现的将语言转换成超图表示的抽象逻辑形式的自然语言理解流程框架,借助链语法输出的链集合,将其映射到使用超图来表示的更抽象的逻辑关系集合。

第六章,介绍本文设计实现的将超图表示的抽象逻辑形式转换自然语言语句的自然语言生成流程框架,通过微观规划器来重新组织和整理超图表示的抽象逻辑关系,并利用基于超图匹配的表层生成器将这些整理后的抽象逻辑关系转换成自然语言语句。

第七章,介绍本文设计的对自然语言语句进行逻辑推理的机制,结合本文设计和实现的自然语言理解框架,将自然语言语句转换成超图形式表示的抽象逻辑关系集合,并使用概率逻辑网络进行推理,最终将推理后得到的超图表示的逻辑关系集合转换成自然语言语句的形式,特别给出了对带比较级的句子进行推理的过程。

第八章,结合第五、六、七章的研究成果,将简单的英语维基百科上的知识表示成第二章中的超图形式,并在第四章的动机驱动的对话控制模型上,设计并实现了一个以信息查询为动机的基于概率逻辑推理的问答系统。

第九章,对全篇论文的研究进行总结,列出本文的主要贡献和创新点,并提出下一步的研究方向和对未来研究的展望。



%%%%%%%%%%%%%%%%%%%%%%%%%%%%%%%%%%%%%%%%%%%%%%%%%%%%%%%%%%%%%%%
最后一章结构:
1)本文的主要贡献与创新
2)下一步研究方向
3)前景与展望
%%%%%%%%%%%%%%%%%%%%%%%%%%%%%%%%%%%%%%%%%%%%%%%%%%%%%%%%%%%%%%%
XXXXX


%%%%%%%%%%%%%%%%%%%%%%%%%%%%%%%%%%%%%%%%%%
\cite{Monteleone2013} rejected

\cite{Libkin2001}

@article{Libkin2003,
 author = {Libkin, Leonid},
 title = {Expressive Power of SQL},
 journal = {Theor. Comput. Sci.},
 issue_date = {14 March 2003},
 volume = {296},
 number = {3},
 month = mar,
 year = {2003},
 issn = {0304-3975},
 pages = {379--404},
 numpages = {26},
 url = {http://dx.doi.org/10.1016/S0304-3975(02)00736-3},
 doi = {10.1016/S0304-3975(02)00736-3},
 acmid = {782738},
 publisher = {Elsevier Science Publishers Ltd.},
 address = {Essex, UK},
 keywords = {SQL, aggregation, databases, expressive power, locality, query languages},
} 


\cite{Liang2011}

\cite{Bollacker2008}