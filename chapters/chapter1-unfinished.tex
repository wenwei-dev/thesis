
%%%%%%%%%%%%%%%%%%%%%%%%%%%%%%%%%%%%%%%%
DSC树部分

\cite{Liang2013}have suggested reducing the task to a simpler
problem of constraint satisfaction, and by that allowed for efficient inference
to be carried out. They developed a new representation called DCS trees – dependency
based compositional semantics trees – and map utterances to these
trees in an efficient way through the constraints of pre-specified trigger words
and a predicate dictionary given in advance (for example the word city would
trigger the predicate city). Although the model was assessed on small scale
data-sets with some limitations for scalability originating from the specification
of trigger words, it can easily be scaled-up to map sentences over much larger
domains through the use of role-labelled input and shallow extraction of predicates
from text. Together with recently published large scale fact data-sets such
as Freebase\cite{Bollacker2008} this model
will allow for scalable question answering to be carried out extending on existing
question answering systems that rely on simple role-labelling and empty slot
completion based on previously observed statements.

However the developed representation, studied thoroughly in \cite{Liang2011},
lacks a formal proof for the expressive power it offers. It is unknown yet whether
all sentences expressible in first order logic can be captured using DCS trees and
their possible extensions. Furthermore, the formulation of DCS is analogous to
and inspired by that of database querying languages (from personal communication
with Liang). But even for these well studied database querying languages
it has been shown that the expressiveness offered is very limited \cite{Libkin2003}.
It has been shown that small differences in the language definition affect the
behaviour of the language dramatically and that these languages cannot define
recursive queries regardless of the aggregate functions and arithmetic operations
specified (these recursive structures can, however, be observed in natural language).
Furthermore, it was shown that the problem of proving expressiveness
bounds for these is as hard of a problem as some long-standing open problems
in complexity theory\cite{Libkin2003}.

The use of a database querying languages for the representation of natural language utterances and not just knowledge by itself is a very interesting idea that allows for efficient logical inference and entailment to be carried out in the form of constraint satisfaction, and indeed has been studied before
\cite{Giordani2010a}; 
\cite{Giordani2010b}, \cite{Giordani2009}. However, even when justified with the
explicit assumption placed in advance of capturing only a subset of language
utterances, there is still the question of which representation to choose. The
mapping of utterances to the SQL query language offers the advantages of the
well studied language1
. The expressive power of DCS is still an open problem
to be explored.


%%%%%%%%%%%%%%%%%%%%%%%%%%%%%%%%%%%%%%%%%%

\cite{Monteleone2013} rejected

\cite{Libkin2001}

@article{Libkin2003,
 author = {Libkin, Leonid},
 title = {Expressive Power of SQL},
 journal = {Theor. Comput. Sci.},
 issue_date = {14 March 2003},
 volume = {296},
 number = {3},
 month = mar,
 year = {2003},
 issn = {0304-3975},
 pages = {379--404},
 numpages = {26},
 url = {http://dx.doi.org/10.1016/S0304-3975(02)00736-3},
 doi = {10.1016/S0304-3975(02)00736-3},
 acmid = {782738},
 publisher = {Elsevier Science Publishers Ltd.},
 address = {Essex, UK},
 keywords = {SQL, aggregation, databases, expressive power, locality, query languages},
} 


\cite{Liang2011}

\cite{Bollacker2008}