一个能拟人化的认知对话系统通常要求从多个句子甚至不同的对话中获取相关联的信息,这不仅要求系统的知识表示形式能允许持续存储对应的信息以及关联信息,还要求系统能在这样的知识表示库上进行灵活高效的语义知识操作。

上节我们看到的例子是FCG使用的基于逻辑的语义表达形式。在本文的研究中,我们将使用一种不同的逻辑表达形式,即OpenCog中概率逻辑网PLN(Probabilistic Logic Networks)的形式体系。

\begin{enumerate}
\item {逻辑知识表示的优缺点}

根据Pei Wang \cite{Wang2006}的理论,总体来说,逻辑推理系统通常包括以下组成部分:

\begin{itemize}
\item 一种形式化语言, 能用于知识表示,以及系统与其环境之间的沟通。
\item 一个语义系统,用于决定词的意义和句子真值。
\item 一套推理规则,匹配问题和知识,能从前提推出结论等。
\item 一个存储器,可以储存问题和知识,并提供推理的工作空间。
\item 一个控制机制,负责选择前提和每一步推理中需要的推理法则。
\end{itemize}

\noindent 前3个通常被认为是逻辑,或推理系统中的逻辑部分,最后的2个则被认为是推理系统中的控制部分。

使用逻辑来表达自然语言的语义,这涉及精确度和灵活度之间的平衡。逻辑是精确的,它的标志是“确定”。它带来一种用作定理证明的思考方式。它的优势在于稳定和系统的方式对表达式赋值并保持表达式的真值。在几乎没有歧义的技术领域,精确度是非常重要的,而逻辑显然是一个极好的架构。但是,当逻辑在意思和真值都比较模糊和模棱两可的领域中应用时,其适用性就不那么明显了,而且在人工智能和NLP领域中,逻辑的应用也常常引起争议。

进一步来说,逻辑领域中最长和最丰富的传统是以演绎推理为中心的。演绎推理是有限的推理形式,而且人们很少做关于自然语言的常识性推理,但仍然有大量的“归纳逻辑”\cite{Muggleton1994} \cite{Holland1989}工作(包括最近人们重视的“概率归纳逻辑编程” \cite{Riguzzi2014}和溯因推理\cite{Queiroz2005} \cite{Menzies1996})。这些推理方法也大量存在于我们所用的PLN系统中。

%%%%%%%%%%%
\item {谓词逻辑 VS 传统逻辑}
%%%%%%%%%%%

在数学和NLP环境下,最普遍的逻辑形式是“谓词逻辑”。它的独特之处是对变量的使用,这些变量可以被量化。两个常见的限量词是存在量词$\exists$(“存在”)和普遍量词$\forall$(“所有”)。在谓语逻辑公式中,变量可能是人们谈及的宇宙元素,也许是超越宇宙的关系或函数。举例来说:在标准的谓语逻辑中,一个句子,例如“Ravens are black” (乌鸦是黑的),它呈现的是个“一般命题”,如:
$$
(\forall x)(Raven(x) \rightarrow Black(x)).
$$

谓语逻辑通常与语义学的“理论模型”法一同出现,它视“逻辑公式”为特定领域的模型\cite{Muller2009}。


另一个方法是“传统逻辑”推理法,这个概念实际上可以追溯到亚里士多德。在这里,“基本假设”是:命题由两个词语构成,推理过程反过来根据命题来构建。详解如下:

\begin{itemize}
\item 一个“词语”是言语表达的一部分,但就其本身而言,并没有对或错,例如“男人”或“凡人”。
\item 一个“命题”包含两个词语,其中一个(“谓语”)是对其它词(“主语”)的“证实”或“否认”。它能够反应“真”或“假”。
\item “三段论”是一种推理法,其中的一个命题(“结论”)必须遵循另外两个(“前提”)。
\end{itemize}

在“传统逻辑”推理法中,我们可以这样推理:
\begin{eqnarray*}
raven \rightarrow black
\end{eqnarray*}
无需引入量词。

以下是一个标准的“三段论”逻辑推理示例:
\begin{eqnarray*}
A \rightarrow B \\
B \rightarrow C \\
\Rightarrow \\
A \rightarrow C
\end{eqnarray*}
这也是演绎推理的简单形式。

%%%%%%%%%%%
\item{目前的相关系统}
%%%%%%%%%%%

\begin{description}

\item[(1)] NARS

在人工智能领域,Pei Wang的NARS引擎运用了传统逻辑推理法\cite{Wang2006},并引入了独特的数学运算,用于管理与“传统逻辑关系”有关的不确定性。NARS是建构在经验的基础上,而不是模型论语义学。

在许多方面,我们所使用的PLN逻辑形式化体系与NARS有类似的地方,但也存在巨大差异。PLN在一个独特的数学框架下,同时利用传统逻辑和谓语逻辑。此外,PLN还使用概率数学,以推导不确定真值的公式。反之,NARS是基于原始的、非概率的、不确定的管理体系。PLN也是以经验语义为基础,但形式与NARS不同。


图\ref{fig:nars}展示了基本演绎推理、归纳推理和外展推理公式。这些公式是PLN和NARS共有的。在每个关系式右边的$<s,c>$表示“每个关系的优势和信心”。PLN和NARS使用不同的公式,从那些前提中推导(优势、信息)结论的真值。

\begin{figure}[htb]
\centering
\includegraphics[width=12cm]{figures/nars.png}
\caption{ NARS/PLN 传统逻辑中演绎推理、归纳推理和回溯推理的形式 }
\label{fig:nars}
\end{figure}

\item[(2)]  Cyc

在NLP系统中,开发最彻底的“谓词逻辑”要属由Cycorp公司开发的商用系统Cyc\cite{Lenat1990}。Cyc系统拥有一个非常丰富的人工编码知识库,它的终极目标是:以谓词逻辑的形式,将所有人类常识进行编码。虽然它的知识库中已存有数百万的谓语逻辑公式,但到目前为止,似乎只对一小部分人类常识进行了编码。 不过,作为智能应用系统,Cyc已经相当成功了。

我们之所以在这里讨论Cyc“知识表示”的基础,是为了与我们的系统进行比较。在Cyc系统中,概念名称被称作“常量”。在书写时,它们以"\#\$"开始。以下是几种常量:

\begin{itemize}
\item 单个词汇被称为“个体词”,例如:\#\$BillClinton (比尔·克林顿 或 \#\$France(法国)。
\item 集合词,例如\#\$Tree(树)-ThePlant(植物) (包括所有的树) 或 \#\$EquivalenceRelation (等价关系)(包括所有的等价关系)。
\item 真值函数,可以运用到一个或多个概念中,并反馈“真”或“假”。例如:\#\$siblings is the sibling relationship,如果两个参数是siblings,那就是真的。按照惯例,真值函数以小写字母开头。真值函数可能会被分拆为逻辑连接词(如:\#\$and, \#\$or, \#\$not, \#\$implies)和量词(如:\#\$forAll, \#\$thereExists等)。
\item “函数”能够从给定的词中生成新词汇,例如:
\#\$FruitFn,当提供了一个描述某种植物类型(或集合)的参数,它会给出这类植物的一组水果。按照惯例,“函数常量”以“大写字母”开始,以字符“Fn”结束。
\end{itemize}

“常量”之间由谓语联系在一起。最重要的谓语是: \#\$isa 和 \#\$genls。第一个(\#\$isa)描述的是:某个个体是某个集合中的一个例子(如:specialization);第二个(\#\$genls)描述的是:一个集合是另一个集合的子集合(例如:generalization)。
利用特定的CycL句子,我们可以得出概念事实。谓语是写在它的参数之前的,在圆括号内。例如:

 {\tt\begin{small}\begin{lstlisting}
    (\#\$isa \#\$BillClinton \#\$UnitedStatesPresident) \;
    \end{lstlisting}\end{small}}

\noindent 意思是 "Bill Clinton belongs to the collection of U.S. presidents" ;

 {\tt\begin{small}\begin{lstlisting}
    (\#\$genls \#\$Tree-ThePlant \#\$Plant) \;
    \end{lstlisting}\end{small}}

\noindent 意思是 "All trees are plants"; and

 {\tt\begin{small}\begin{lstlisting}
    (\#\$capitalCity \#\$France \#\$Paris) \;
    \end{lstlisting}\end{small}}

\noindent 意思是 "Paris is the capital of France."


下面是比较封复杂的例子,它表示了一组或一类词的规则,而不是任意特定的个别词:

 {\tt\begin{small}\begin{lstlisting}
 (\#\$relationAllExists \#\$biologicalMother
 \#\$ChordataPhylum \#\$FemaleAnimal)
    \end{lstlisting}\end{small}}


Cyc知识库被分为多个“微理论”库,它们是概念和事实的集合,每个“微理论”库都与一个特定领域相关联。每个“微理论”库都不能有相互矛盾的信息,而且可以通过Bayesian网络提供概率真值。

Cyc配备了一个复杂的、基于短语结构语法的NLP系统,这个系统将自然语言句子映射为Cyc逻辑形式。由于这个系统本身的特性,我们尚不清楚它到底具有什么样的优点和缺点。


\item[(3)] 概念网ConceptNet

知识表示的另一个模式是由ConceptNet提供的\cite{Liu2004}。它从Cyc系统和形式逻辑中借鉴了一些理念,但没有考虑复杂的位元,如量词,并且比较接近自然语言层面。ConceptNet 是个大规模的语义网络,它表达概念(由单词或短语表述)之间的关系。图\ref{fig:concept}是关于子网络的例子。

\begin{figure}[htb]
\centering
\includegraphics[width=12cm]{figures/conceptnet.png}
\caption{ An illustrative fragment of ConceptNet }
\label{fig:concept}
\end{figure}

ConceptNet系统的节点是自然语言碎片,这些碎片按照某个句法模式的本体进行了半结构化。它们适合一阶概念(作为名词短语,如potato chips)和二阶概念(作为动词短语,如:buy potato chips)。

节点可以由19个语义关系相连接:
\begin{itemize}

\item 事物
\begin{itemize}
\item IsA ?(corresponds loosely to hypernym in WordNet)
\item PropertyOf ?(e.g. (PropertyOf ``apple'' ``healthy''))
\item PartOf ?(corresponds loosely to holonym in WordNet)
\item MadeOf ?(e.g. (MadeOf ``bottle'' ``plastic''))
\end{itemize}

\item 事件
\begin{itemize}
\item FirstSubeventOf, LastSubeventOf ?(e.g. (FirstSubeventOf ``act in play'' ``learn script''))
\item EventForGoalEvent ?(e.g. (EventForGoalEvent ``drive to grocery store'' ``buy food''))
\item EventForGoalState ?(e.g. (EventForGoalState ``meditate'' ``enlightenment''))
\item EventRequiresObject ?(e.g. (EventRequiresObject ``apply for job'' ``resume''))
\end{itemize}

\item 动作
\begin{itemize}
\item EffectOf ?(e.g. (EffectOf ``commit perjury'' ``go to jail''))
\item EffectOfIsState ?(e.g. (EffectOfIsState ``commit perjury'' ``criminal prosecution''))
\item CapableOf ?(e.g. (CapableOf ``police officer'' ``make arrest''))
\end{itemize}

\item 空间
\begin{itemize}
\item OftenNear ?(e.g. (OftenNear ``sailboat'' ``marina''))
\item LocationOf ?(e.g. (LocationOf ``money'' ``in bank account''))
\end{itemize}

\item 目标
\begin{itemize}
\item DesiresEvent, DesiresNotEvent ?(e.g. (DesiresEvent ``child'' ``be loved''))
\end{itemize}

\item 功能
\begin{itemize}
\item UsedFor ?(e.g. (UsedFor ``whistle'' ``attract attention''))
\end{itemize}

\item 通用
\begin{itemize}
\item CanDo ?(e.g. (CanDo ``ball'' ``bounce''))
\item ConceptuallyRelatedTo ?(e.g. (ConceptuallyRelatedTo ``wedding'' ``bride'' ``groom'' )
\end{itemize}
\end{itemize}

\end{description}