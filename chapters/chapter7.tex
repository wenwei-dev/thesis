

%%%%%%%%%%%%%%%%%%%%%%%%%%%%%%%%%%%%%%%%%%%%%%%%%%%%%%%%%%%%%%%
\chapter{基于超图表示的语言逻辑推理的设计与实现}{Implementation: Language Reasoning on Hypergraph Knowledgebase}
\label{chap:reasoning}
%%%%%%%%%%%%%%%%%%%%%%%%%%%%%%%%%%%%%%%%%%%%%%%%%%%%%%%%%%%%%%%

基于自然语言的逻辑推理在很多应用领域都起到了非常重要的作用,其中也包括本文第 \ref{chap:dialogue}章将介绍的智能会话系统。针对第\ref{chap:representation}章阐述的基于超图的知识表示以及逻辑推理系统等的理论知识,并在第 \ref{chap:comprehension}章介绍的自然语言理解系统的基础上,我们进一步设计并实现了一个基于超图表示的语言逻辑推理系统。本章将通过几个典型例子来介绍该语言逻辑推理系统的设计思路和实现步骤,也就是将自然语言通过自然语言理解系统转换成超图的表示形式,并使用概率逻辑网PLN来进行一些常识推理的过程。


%%%%%%%%%%%%%%%%%%%%%%%%%%%%%%%%%%%%%%%%%%%%%%%%%%%%%%%%%%%%%%%
\subsection{有关比较级的推理}{Reasoning About Comparatives}
%%%%%%%%%%%%%%%%%%%%%%%%%%%%%%%%%%%%%%%%%%%%%%%%%%%%%%%%%%%%%%%

首先,紧接着上一章中对比较级的句法分析的讨论,我们这里通过几个例子来展示,在我们的自然语言系统输出的有关比较级的逻辑表达式上,PLN如何实现相关推理。为了更好地显示有效的推理过程,我们在下面的例子中过滤了和比较级推理无关的RelEx关系和逻辑表达。

假设有:

\begin{itemize}
\item Bob likes Hendrix more than the Beatles
\item Bob is American
\item Menudo is liked less by Americans than the Beatles
\end{itemize}

\noindent 那么我们可以推出:Bob likes Hendrix more than Menudo.

对于第一个句子,通过执行RelEx和RelEx2Logic,可以得到:

 {\tt\begin{small}\begin{lstlisting}
_subj(like, Bob)
_obj(like, Hendrix)
than(Hendrix, Beatles)
_comparative(like, Hendrix)
==>
TruthValueGreaterThanLink
    EvaluationLink like Bob Hendrix
    EvaluationLink like Bob Beatles
\end{lstlisting}\end{small}}

\noindent 类似地,第二个句子:

 {\tt\begin{small}\begin{lstlisting}
_subj(like, Americans)
_obj(like, Menudo)
than(Beatles, Menudo)
_comparative(like, Beatles)
==>
TruthValueGreaterThanLink
    EvaluationLink like Americans Beatles
    EvaluationLink like Americans Menudo
\end{lstlisting}\end{small}}

从这些句子中得到的逻辑表达是非常透明的。简单地根据“TruthValueGreaterThan”关系是可传播,以及“Bob is American”,PLN很容易推理得到结论:

 {\tt\begin{small}\begin{lstlisting}
TruthValueGreaterThanLink
    EvaluationLink like Bob Hendrix
    EvaluationLink like Bob Menudo
  \end{lstlisting}\end{small}}
  
那么现在所处理的知识是逻辑形式而非句法形式,因此可以调用知识库中的相关知识加入到推理过程中。例如,假设我们还知道“Bob likes Sinatra more than Menudo”,可表示如下:
    
     {\tt\begin{small}\begin{lstlisting}
TruthValueGreaterThanLink
    EvaluationLink like Bob Hendrix
    EvaluationLink like Bob Menudo
  \end{lstlisting}\end{small}}
  
  
  \noindent 根据PLN中的回溯推理规则 $\left((A\rightarrow C) \wedge (B\rightarrow C) \Rightarrow(A\rightarrow C)\right)$,可以得到:
  
     {\tt\begin{small}\begin{lstlisting}
SimilarityLink
    Hendrix
    Sinatra
  \end{lstlisting}\end{small}}
  
 

%%%%%%%%%%%%%%%%%%%%%%%%%%%%%%%%%%%%%%%%%%%%%%%%%%%%%%%%%%%%%%%
\subsection{基于自然语言的三段论推理}{Syllogistic Reasoning From Natural Language Premises}
%%%%%%%%%%%%%%%%%%%%%%%%%%%%%%%%%%%%%%%%%%%%%%%%%%%%%%%%%%%%%%%

为了测试RelEx2Logic的输出用于逻辑推理的有效性,我们使用了PLN来执行基于自然语言输入的三段论的推理。接下来,我们给出一个基于自然语言的三段论推理的例子,和详细的推理步骤。


假设我们需要完成的推理如下:

\begin{verbatim}
Socrates is a man.(苏格拉底是人)
Men breath air.(人呼吸空气)
|- 
Socrates breath air.(苏格拉底呼吸空气)
\end{verbatim}

对于此例,推理前提在Atomspace中可表示如下:

{\tt\begin{small}\begin{lstlisting}
Concepts

(ConceptNode "Socrates@f250f429-5078-4453-8662-c63cc8f58a22") ; [217]

(ConceptNode "Socrates" (stv .1 1.0)) ; [218]

(ConceptNode "man@80d4e852-1b93-4c49-84e1-5a7352b0dcb1" (stv .1 1.0)) ; [220]

(ConceptNode "men@83d83a2e-940c-46aa-91f1-a7d2c106ef13" (stv .1 1.0)) ; [290]

(ConceptNode "man" (stv .1 1.0)) ; [221]

(ConceptNode "air@e3f175ea-c3d1-45cf-883d-a6d2f6a879ac") ; [292]

(ConceptNode "air") ; [293]

(ConceptNode "present") ; [227]

满足breathe(x,y)的多元组可表示如下:

(ListLink (stv 1.000000 0.000000)
  (ConceptNode "men@83d83a2e-940c-46aa-91f1-a7d2c106ef13") ; [290]
  (ConceptNode "air@e3f175ea-c3d1-45cf-883d-a6d2f6a879ac") ; [292]
) ; [295]

breathe(x,y)

(EvaluationLink (stv 1.000000 1.000000)
  (PredicateNode "breathe@218b15b2-52a8-430c-93f6-b4bba225418c") ; [287]
  (ListLink (stv 1.000000 0.000000)
    (ConceptNode "men@83d83a2e-940c-46aa-91f1-a7d2c106ef13") ; [290]
    (ConceptNode "air@e3f175ea-c3d1-45cf-883d-a6d2f6a879ac") ; [292]
  ) ; [295]
) ; [296]

"Socrates IS-A man"(“苏格拉底是人”)可表示如下:

(InheritanceLink (stv 1.000000 1.000000)
  (ConceptNode "Socrates@f250f429-5078-4453-8662-c63cc8f58a22") ; [217]
  (ConceptNode "man@80d4e852-1b93-4c49-84e1-5a7352b0dcb1") ; [220]
) ; [223]

具体实例继承于泛化概念:

(InheritanceLink (stv 1.000000 0.990000)
  (ConceptNode "men@83d83a2e-940c-46aa-91f1-a7d2c106ef13") ; [290]
  (ConceptNode "man") ; [221]
) ; [291]

(InheritanceLink (stv 1.000000 1.000000)
  (ConceptNode "Socrates@f250f429-5078-4453-8662-c63cc8f58a22") ; [217]
  (ConceptNode "Socrates") ; [218]
) ; [219]

(InheritanceLink (stv 1.000000 0.990000)
  (ConceptNode "man@80d4e852-1b93-4c49-84e1-5a7352b0dcb1") ; [220]
  (ConceptNode "man") ; [221]
) ; [222]
 \end{lstlisting}\end{small}}
 

下图显示了PLN的推理路径以及每一步中使用的推理规则:

{\tt\begin{small}\begin{lstlisting}
1) 成员变量泛化规则(GeneralEvaluationToMemberRule)
输入:
(EvaluationLink (stv 1.000000 1.000000)
  (PredicateNode "breathe@7f5b37e8-e4b3-4335-a06b-68af470cf354") ; [350]
  (ListLink (stv 1.000000 0.000000)
    (ConceptNode "men@a2905bdd-9214-4717-82c6-dfe21c1263bc") ; [353]
    (ConceptNode "air@9bfe790d-5446-4219-be74-845c0d145fe1") ; [355]
  ) ; [358]
) ; [359]

输出:(以及ListLink中其他变量被泛化后的3个其他变体)
(MemberLink (stv 1.000000 1.000000)
  (ConceptNode "men@a2905bdd-9214-4717-82c6-dfe21c1263bc") ; [353]
  (SatisfyingSetLink (stv 1.000000 1.000000)
    (VariableNode "$X0") ; [441]
    (EvaluationLink (stv 1.000000 0.000000)
      (PredicateNode "breathe@7f5b37e8-e4b3-4335-a06b-68af470cf354") ; [350]
      (ListLink (stv 1.000000 0.000000)
        (VariableNode "$X0") ; [441]
        (ConceptNode "air@9bfe790d-5446-4219-be74-845c0d145fe1") ; [355]
      ) ; [443]
    ) ; [444]
  ) ; [445]
) ; [446]

2) 成员继承规则(MemberToInheritance Rule)
输入:上一步的输出

输出:
(InheritanceLink (stv 1.000000 1.000000)
  (ConceptNode "men@a2905bdd-9214-4717-82c6-dfe21c1263bc") ; [353]
  (SatisfyingSetLink (stv 1.000000 1.000000)
    (VariableNode "$X0") ; [441]
    (EvaluationLink (stv 1.000000 0.000000)
      (PredicateNode "breathe@7f5b37e8-e4b3-4335-a06b-68af470cf354") ; [350]
      (ListLink (stv 1.000000 0.000000)
        (VariableNode "$X0") ; [441]
        (ConceptNode "air@9bfe790d-5446-4219-be74-845c0d145fe1") ; [355]
      ) ; [443]
    ) ; [444]
  ) ; [445]
) ; [6107]

3) 溯因推理规则(AbductionRule)
输入:
(InheritanceLink (stv 1.000000 0.990000)
  (ConceptNode "men@a2905bdd-9214-4717-82c6-dfe21c1263bc") ; [353]
  (ConceptNode "man") ; [284]
) ; [354]

(InheritanceLink (stv 1.000000 0.990000)
  (ConceptNode "man@fbc51aff-8074-46d8-b3ba-1ccaeb1adb34") ; [283]
  (ConceptNode "man") ; [284]
) ; [285]

输出:
(InheritanceLink (stv 1.000000 0.988901)
  (ConceptNode "man@fbc51aff-8074-46d8-b3ba-1ccaeb1adb34") ; [283]
  (ConceptNode "men@a2905bdd-9214-4717-82c6-dfe21c1263bc") ; [353]
) ; [609]

4) 演绎推理规则(DeductionRule)
输入:上一条输出以及下面的原子集合
(InheritanceLink (stv 1.000000 1.000000)
  (ConceptNode "Socrates@46ec3d0f-4535-4d01-87b7-84ef65c25a23") ; [280]
  (ConceptNode "man@fbc51aff-8074-46d8-b3ba-1ccaeb1adb34") ; [283]
) ; [286]

输出:
(InheritanceLink (stv 1.000000 0.991755)
  (ConceptNode "Socrates@46ec3d0f-4535-4d01-87b7-84ef65c25a23") ; [280]
  (ConceptNode "men@a2905bdd-9214-4717-82c6-dfe21c1263bc") ; [353]
) ; [777]

5) 演绎推理规则(DeductionRule)
输入:步骤2)和步骤4)的输出

输出:
(InheritanceLink (stv 1.000000 0.986333)
  (ConceptNode "Socrates@46ec3d0f-4535-4d01-87b7-84ef65c25a23") ; [280]
  (SatisfyingSetLink (stv 1.000000 0.000000)
    (VariableNode "$X1") ; [442]
    (EvaluationLink (stv 1.000000 1.000000)
      (PredicateNode "breathe@7f5b37e8-e4b3-4335-a06b-68af470cf354") ; [350]
      (ListLink (stv 1.000000 0.000000)
        (VariableNode "$X1") ; [442]
        (ConceptNode "air@9bfe790d-5446-4219-be74-845c0d145fe1") ; [355]
      ) ; [922]
    ) ; [923]
  ) ; [8605]
) ; [12317]

6) 成员变量继承规则(InheritanceToMemberRule)
输入:上一条规则的输出

输出:
(MemberLink (stv 1.000000 0.989841)
  (ConceptNode "Socrates@46ec3d0f-4535-4d01-87b7-84ef65c25a23") ; [217]
  (SatisfyingSetLink (stv 1.000000 1.000000)
    (VariableNode "$X0") ; [385]
    (EvaluationLink (stv 1.000000 0.000000)
      (PredicateNode "breathe@7f5b37e8-e4b3-4335-a06b-68af470cf354") ; [350]
      (ListLink (stv 1.000000 0.000000)
        (VariableNode "$X1") ; [442]
        (ConceptNode "air@9bfe790d-5446-4219-be74-845c0d145fe1") ; [355]
      ) ; [922]
    ) ; [388]
  ) ; [389]
) ; [6375]

7) 成员变量赋值规则(MemberToEvaluationRule)
输入:上一条规则的输出

最终推理结论:
(EvaluationLink (stv 1.000000 0.989841)
  (PredicateNode "breathe@7f5b37e8-e4b3-4335-a06b-68af470cf354") ; [350]
  (ListLink (stv 1.000000 0.000000)
    (ConceptNode "Socrates@46ec3d0f-4535-4d01-87b7-84ef65c25a23") ; [217]
    (ConceptNode "air@9bfe790d-5446-4219-be74-845c0d145fe1") ; [355]
  ) ; [922]
) ; [388]
 \end{lstlisting}\end{small}}

\noindent 需要说明的是,上述提供的从前提到结论的推理路径中,只显示了在推理过程中提供有用信息的推理步骤。在推理过程中,PLN的后向推理链尝试并放弃大量其他的可能推理步骤。


%%%%%%%%%%%%%%%%%%%%%%%%%%%%%%%%%%%%%%%%%%%%%%%%%%%%%%%%%%%%%%%
\section{本章小结}{Summary of Accomplishments and Future Work}
%%%%%%%%%%%%%%%%%%%%%%%%%%%%%%%%%%%%%%%%%%%%%%%%%%%%%%%%%%%%%%%

本章介绍了几个使用自然语言理解系统的输出在PLN上的推理实验。 实验表明,我们的自然语言理解系统输出的逻辑形式能和逻辑推理引擎结合,并能进行简单的常识推理。但是目前的系统还无法完成较复杂的推理。我们下一步将继续改进RelEx2Logic系统,使其能更好地为复杂的句子提供更准确的逻辑表达,并改善PLN的控制机制,使其能进行更复杂的推理。
