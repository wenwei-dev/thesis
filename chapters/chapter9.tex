\chapter{总结与展望}{Conclusion: Accomplishments and Future Research}
\label{chap:conclusion}
XXXX这章需要重新写

实现一个完全达到人类智能标准的智能对话系统无疑是一个精彩的挑战,但在当前的科学技术水平下,也无疑是一个非常困难的研究项目。本文针对目前主流的基于规则或者基于统计的对话系统研究中的存在的问题,提出了一个XXXXX

%%%%%%%%%%%%%%%%%%%%%%%%%%%%%%%%%%%%%%%%%%%%%%%%%
\section{本文的主要贡献和创新}{Accomplishments}
%%%%%%%%%%%%%%%%%%%%%%%%%%%%%%%%%%%%%%%%%%%%%%%%%
XXX本文的主要贡献和创新,以及存在的问题。

%%%%%%%%%%%%%%%%%%%%%%%%%%%%%%%%%%%%%%%%%%%%%%%%%
\section{下一步研究方向}{Future Research}
%%%%%%%%%%%%%%%%%%%%%%%%%%%%%%%%%%%%%%%%%%%%%%%%%
XXX下一步研究方向

我们最初是秉持创造出智能性自然语言对话系统之实务目标,着手进行本文研究的设计和开发。我们探讨的方法论是基于四个假设,如前言所述:

\begin{itemize}
\item {\bf 假设1——关于自然语言理解:}借助一个超图转换系统,使用依存关系语法、传统逻辑与谓词逻辑的合理结合,将各式各样的自然语言表达转换成满足下列要求的逻辑表达方式,是可行的。
    \begin{itemize}
    \item 包含自然语言表达式中的主要语义。
    \item 具体化自然语言表达式中存在的任何无法在语言到逻辑的转换消除的歧义,使得这些歧义能通过基于语境知识的逻辑推理后很直截了当地得到消除。
    \end{itemize}
\item {\bf 假设2——关于基于语言的推理:}借助由超图转换表示的推理规则和基于超图表示的知识库,使用简单的逻辑推理,在上述自然语言理解框架输出的逻辑表达式上实现基本的常识推理,是可行的
\item {\bf 假设3——关于自然语言生成:} 借助一个超图转换系统和一个超图匹配系统,在由自然语言理解系统自动生成的二元组(语言表达式,逻辑表达式)组成的知识库中根据逻辑表达式找到匹配的语言表达式并生成自然语言,是可行
\item  {\bf 假设4——关于智能会话系统:} 利用上述的语言理解、生成和逻辑推理系统,构建一个有用且灵活的智能会话系统,是可行的。
\end{itemize}

在本文研究的六年之中发现,为了创造真正可行的智能会话系统,首先必须对自然语言 (NL) 的理解和生成投注大量心力。最终我们大部分的研发时间都耗费在创造足够广泛、强建稳固的自然语言理解系统,这后来作为我们自然语言生成系统的基础(通过经由反向理解达到完善化阶段语言生成)。然而,我们进行理解、产生和推理的试验,帮助我们透彻地思考设计对话系统的问题。我们在此文呈现的 CogDial 设计,远比初步的设计和现有的原型实现还要充实、缜密,虽然功能有限,但这让我们相信整体设计上是可行的。

本文的研究成果包含:

\begin{itemize}
\item 修复和改进链语法分析器和 RelEx 系统的诸多方面,并设计和实现了 RelEx2Logic 系统,使得自然语言能转换成逻辑形式。因此论证了假设1的正确性
\item 论证演绎推理和针对比较级的推理;PLN 使用RelEx2Logic的输出来作为推理前提,能完成一定的常识推理,因此至少在单纯的案例中论证了假设2的正确性
\item 设计并实行了微观规划器Microplanner和能将基于超图的逻辑表达转换成英文句子的表层生成器SuReal,通过反向操作我们的自然语言理解通道 (NL comprehension pipeline),使得逻辑形式能转换成自然语言,从而架起了语言和逻辑之间的桥梁,也因此论证了假设3的正确性
\item 根据言语行为理论和 OpenCog 框架,结合情感计算模型OpenPsi中的目标驱动的机制,提出了一个智能会话系统的具体设计;并使用设计的核心概念来实行一个原型问答系统,进而证明此设计的可行性。这提供了假设4可能属实的初步迹象,但仍有许多工作需要完成。 
\item 提出了无监督语言学习的概念设计,为使用通过机器学习的规则来取代我们目前自然语言理解系统中使用的人工编写的规则库奠定了基础。

\end{itemize}

要实现创造出具有充分能力的智能会话系统之宏伟目标,仍有大量的研究工作待完成。但我们已在人工智能对话的每一个关键层面有实质上的进展,对于逻辑对话系统的每个元件如何运作、以及这些元件应如何一起运行,我们都学到了不少。我们也创造出有价值的软件工具,这些都正由开放源社区的其他人员积极的改良和发展,最终很有可能被充分发挥、利用在实际产品。在许多不同层面上,我们完成的研究为往后更多智能对话系统的研发打下了实用与概念的基础。
