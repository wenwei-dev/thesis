\chapter{前言}{Preface}

目前,计算语言学软件几乎在每个关键领域都远远落后于人类的认知,如自然语言理解、生成和对话。计算语言学系统的功能在近几十年来得到稳步提高,在很大程度上是由于互联网的总体发展和相关的“大数据”现象所带来的语料库语言学的兴起。然而,在目前这么多研究方向中,哪一个最能带领该领域快速持续发展,还没有定论。

本文提出的结论,是遵循计算语言学功能发展的一个特定方向。本文工作是在一个更大的项目中进行的,该大项目旨在宏观自然语言理解、生成和对话的长期目标。然而,在特定的成果方面,本文也站在自己的立场在大项目的构想里提供了具有独立价值的特定成果。

本文的研究基于三个主要假设。基于OpenCog通用智能系统,我们分别在基于权重的知识表示、标记超图以及超图转换的认知运算方面提出了假设。这些假设可以简明地总结如下:

\begin{itemize}
\item 假设1——关于自然语言理解:通过一个超图转换系统,将各种各样的自然语言表达式转换成逻辑表达式是可行的。采用依存于法、词项逻辑与谓语逻辑的合理组合,以以下方式产生逻辑表达式:
    \begin{itemize}
    \item 捕捉自然语言表达式的主要语义。
    \item 具体化自然语言表达式中任何的歧义。对那些在语言逻辑转换过程中未解决的歧义,用相对直接了当的方式,基于上下文知识进行逻辑推理。
    \end{itemize}
\item 假设2——关于语言推理:基于上述自然语言理解框架输出的逻辑表达式,使用超图转换的推理规则和人类常识推理,是可行的。
\item 假设3——关于自然语言生成:通过超图转换系统和围绕匹配知识库这样一个过程,将各种各样的逻辑表达式转换为自然语言表达式是可行的。知识库是由通过自动语言理解组成的(语言表达式,逻辑表达式)配对。
\end{itemize}

本文提供了大量的证据支持这三个假设。本文使用软件系统进行了以下转换:英文—逻辑—英文,结果与三个假设一致。在这点上,这些系统不提供完整、全面覆盖的英文。但目前没有一个计算语言学系统可以提供完整、全面覆盖任何一个自然语言。特别的:
\begin{itemize}
\item 对于假设1,我们论证了一个自然语言理解的通道,它将卡耐基-梅隆大学链接解析器改进版本{XX}、RelEx关系提取系统的改进版本{ XX }以及一个全新的RelEx2Logic系统{ XX }——将RelEx的输出映射到OpenCog人工总体智能框架所引用的概率逻辑网络(PLN)形式的逻辑表达式,链接到一起。文献[]中报道过这个框架的早期阶段;本文所述版本更为精准,功能更为强大。
\item 对于假设2,以上述自然语言理解框架所解释的英文句子得到的逻辑表达式的前提下,我们论证了概率逻辑网络(PLN)框架进行各种逻辑推理的例子。本文将PLN推理应用到对自然语言理解的输出,取得了比之前更为专业,brittle的成果。
\item 对于假设3,我们论证了一个新的自然语言生成系统,该系统在OpenCog系统内实现从一个给定的逻辑表达式生成英文句子,通过一个微观规划阶段和随后的表面实现阶段,, where the surface realization process is based on matching fragment of the provided logic expression to fragments of logic expressions produced via comprehension using the above framework.。文献[ xx]提到了该框架的早期阶段,本文所述版本更为精准,功能更为强大。
\item Toward Hypothesis 4, we have done significant theoretical and design work, and initial prototype experimentation, as reported in Chapter ??. This is the focus of our current and near-future research work.
\end{itemize}

\section{手工编码的优缺点}{Strengths and Weaknesses of Hand-Coded Rules}
在上诉三个假设前提下实现的软件系统主要是基于手工编码的语言规则(在理解方向上,生成过程依赖于隐式的理解规则,而不需要额外的规则)。然而,这不是本文所采用方法的全部。事实上,目前正在进行的研究目标是用通过无监督学习语料库练习获得的自动规则替换手工编码规则,这项工作将在第x章中论述。本文使用手工编码规则库,不是把它作为智能化英文处理的最终、全面的基础。我们认为,到最后,这将是一个不可行的方法,因为对于手工编码,所需的规则数目很可能是不可行的。相反,本文使用手工编码规则库,是因为它能将语言处理架构问题从语言学习和语言内容中分离出来。使用手工编码规则库作为一英语语言内容的“工作原型”,我们将架构问题从学习问题中分离出来,并提出一个语言理解和生成的通用且强大的架构。在这个架构内从学习更广泛的功能语言学内容到操作,作为一个独立的问题,将通过第xx章中阐述。

在承认手工编码规则库方法的局限性的同时,值得肯定的是,该方法潜在的语用价值。从一个目标是开发一个能够在相对具体的语境中实现语言理解和/或生成的软件系统的程序开发者的角度来看,一个基于规则的自然语言系统可能是完全足够的,甚至在某些情况下有可能由于它的可预见性和简单性,成为最好的基于学习的系统。这里所描述的软件系统是以专业方式实现和架构的,并可以在现实世界直接使用的应用软件;事实上,该系统的好几个方面已经以这种方式被使用。

\section{OpenCog框架}{The OpenCog Framework}
如上所述,本文工作已在AGI软件框架OpenCog背景下实现。OpenCog在的自然语言处理和数据挖掘领域已被用于商业应用,如文献[ xx]。它也被用于虚拟世界中的控制虚拟代理[ xx]和人形机器人[xx]。

在文献[ xx]概述的“patternist”系统理论概念下,OpenCog以一种统一的体系结构融合了多种人工智能范式,比如不确定逻辑、computational linguistics、evolutionary program learning和connec- tionist attention allocation。包含了这些范式的识别过程通过一种通用的neural-symbolic hypergraph knowledge store,也就是所谓的原子空间(”原子“包括节点和连接,其中后者包括超链接)相互协作。这些过程之间通过相互作用来激发原子空间高级网络结构的自组织现象。后文将对OpenCognitive做更详细讨论,读者也可参考[?],或从 http://opencog.org获得更多内容。

从计算语言学的角度来看,OpenCog提供了以下主要模块:
\begin{itemize}
\item 高度灵活的知识表示框架(原子空间以及其中包含的PLN原子类型),能直观的表达各种各样的句法和语义结构;
\item PLN逻辑推理引擎,可对语言理解的输出进行各种简单的推理;
\item 超图模式匹配和遍历工具(在原子空间实现),提供基本的架构来简化Relex2Logic的开发,完成微观规划和表层实现组件的工作。
\end{itemize}

\section{遗留问题和持续发展}{Omissions and Ongoing Development}
虽然上述工作已经取得了重要成果,但是从一个更广阔的视野来看,仍然还有很多未完成的工作,其中有许多是目前正在研究和发展的课题。

首先必须强调的是,截止到我们目前所做工作为止,语言理解和生成的许多关键方面一直受到冷遇,甚至完全被忽略。如以下几个例子:
\begin{itemize}
\item 形态。链接语法分析器,作为上述理解系统的初始阶段,最近已扩展到在语素及文字水平进行句法分析;但我们还没有将这项工作纳入到我们自己的工作中。
\item 语用学。我们能够按照他们的实际能力对语句生成逻辑表达式进行分析,但是我们并未执行实际的分析。
\item 消歧。我们已在软件框架opencog中实现了词义消歧系统,但还没有集成进来。
\item 名词回指分辨率。在我们的理解系统内,人称代词的照应分辨率已通过各种启发式算法实现,但名词回指却被忽略。
\end{itemize}

目前我们工作的一部分是指出整体框架中的这些缺陷。然而,我们相信即使给出这些疏漏,我们所做的工作足够为以上三个假设提供合理的证据支持。展示处理上述现象能力必然有助于compelling system,但是对于验证本方法和三个假设的有效性并不必要。

本项工作的其中一个主要目的是只能自然语言对话系统的创建。本文针对对话系统,描述了理解、生成和推理等涉及到的相关组件。本文的工作不仅仅包括理解、生成和推理本身,例如:
\begin{itemize}
\item Integration of logical expressions derived from language with logical expression derived from other,补充的信息来源,例如机器人、虚拟现实化[?]或包含领域专业知识的结构化数据库[?];
\item “对话控制程序”的实现,用来从一个动机框架获取信号,并对最近输入的语言、系统知识的选择,和当前的动机提供语言重新反应条件的指导[?]
\item Attentional control of reasoning, so as to enable more complex chains of inference in the face of large amounts of distracting or irrelevant or peripherally relevant information
\end{itemize}

该领域的发展现状将在第??和??章介绍。这项工作目前仅在小案例里面得到了验证,尚有待进一步深入研究。然而, 在理解、生成和推理等方面,已经有很多具体的研究方向  。我们提出三个核心假设的主要原因,是因为它们提供了一种有效的方法,能够用于为一个大型的智能对话体系结构,创建理解、生成和语义推理子系统,以配合上述各方面工作。
