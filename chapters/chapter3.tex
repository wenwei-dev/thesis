\chapter{OpenCog AGI框架}{The OpenCog Artificial General Intelligence Architecture}

在本章我们将介绍通用人工智能体系结构OpenCog的一些关键技术。OpenCog是一个由多个子系统组成的庞大智能认知体系结构。更多有关此系统的技术,可以参考这本一千多页的书籍【XX】。在此我们只列出与本文研究紧密联系的几个关键的子系统,这样有助于理解本文的研究内容,包括前文中提到的自然语言理解、自然语言生成、逻辑推理以及智能会话系统建模等。
\section{ConPrime设计}{The CogPrime Design}

\indent OpenCog框架的核心是一个认知概念框架,称为CogPrime,但它们两者并不完全等同。OpenCog是一个更泛化的框架,用来实现许多特定AI程序以及可能的AGI设计。同时CogPrime可以单独被实现,而不要求被置于OpenCog框架中。一个在OpenCog中特别实现的CogPrime版本被称为OpenCogPrime。OpenCog是作为CogPrime的一个高效的、规划化的实现来设计的。

本节将概述CogPrime,它是一个实现AGI的概念和技术的框架,以一系列理论为基础。CogPrime的具体实现和测试(在OpenCog框架之内)仍然处在一个较浅的阶段,CogPrime的目标是表现出与人类智力性质接近的泛化智能,并最终可以被扩展到更广泛的领域的智能功能(?)。

CogPrime将在《Engineering General Intelligence》一书中有更详尽的描述[??],它将有超过1000页的篇幅,包括附录。本文的目标是以更紧凑的形式列举出其中一些关键点。在此我们略去CopPrime与其他已有的认知框架的比较,读者可以阅读以下文章[??],其中介绍了当前人工大脑与AGI构架的发展形式,其中包含了一个与较早版本的CogPrime框架的比较。

\subsection{认知协同:CogPrime的核心设计概念}{Cognitive Synergy: A Central Design Concept in OpenCog}
\indent CogPrime核心概念基础是以下三点:
\begin{itemize}
  \item 智能取决于系统的整体知识库中的特定高级结构和动力学的涌现;
  \item 我们尚未发现任何可以涌现出这类结构的算法或方案;
  \item 试图通过在一个系统内整合若干个不同的AI算法和框架来实现这种涌现的过程是微妙的,这些算法与框架的整合方式需要仔细地注意,而到目前为止这种整合并没有以正确的方式被执行。
\end{itemize}

人脑表现为许多不同的结构与动力学的整合,以通用的组件按照可感知的认知结构来组合到一起。然而,大脑的算法和结构被进化过程所影响,各部分紧密地联系在一起,互相适应,就像身体的不同器官之间的互相适应一样。这种协作使得整个系统表现出人类水平的泛化智能。我们相信,目前的AI中所缺少的部分是认知协同:不同组件的互相适应形成了整体恰当上的认知架构,在这个过程中,组件间以动态的方式充分地互相协作、彼此间紧密地相互联系在一起,就像人脑和身体的各部分一样,使得整体的架构和动力学涌现出来。这让我们得到CogPrime的核心假设之一:在一个恰当的认知框架和环境中,通过整合多重的符号和亚符号的学习和记忆组件,可以产生出与人类相当甚至更强的稳定的智能。

这类紧密的整合之所以尚未被充分探索,是因为它涉及到不同的层次,需要进行困难的框架和组件算法设计,这种设计应该保有某种面向架构和动力学的视角,使得它处在一个创造了合适的环境的系统中,以显现出架构和动力学。典型的,与不同的认知功能相对应的AI算法和架构已经通过各个研究者社区基于不同的理论规则进行了部署,并针对每个具体的运行环境进行的相应的性能调校。让这些性能迥异的组件共同工作在一起,并形成真正的协作是难以完成的任务。我们相信,以现有技术创造人类等级的AGI的“关键调料”,正是这种协同,而非一些特定的算法、结构、或架构原则。

\subsection{目前的和将来优先的OpenCog应用}{Current and Prior Applications of OpenCog}
为了更好地理解,到目前用来实现CogPrime的具体平台,我们将简略地讨论通过OpenCog系统实现CogPrime架构的部分工作。

OpenCog是一个开源的软件框架,以CogPrime架构的“OpenCogPrime”实现为目标(目前只是部分实现),已被用于自然语言处理和数据挖掘的商业应用。例如,参考[?]中OpenCogPrime的PLN推理以及RelEx语言处理被整合到一起,用来自动化地处理基于从PubMed中收集的信息的生物假设生成(?biological hypothesis generation)。[?]描述了使用OpenCog的MOSES组件进行生物信息分析;它的用途还被延伸到一些未被公开发表的商业应用中,例如财务预测、基因组、市场营销数据分析以及自然语言处理。在最近的相关工作中,OpenCog被用来控制虚拟世界中的虚拟代理[?]。
在2007-2008年间完成的原型工作涉及到使用一个叫OpenPetBrain的OpenCog的变形来控制虚拟世界中的狗。这些OpenCog控制的虚拟狗并没有表现出接近现实中狗的(或人类小孩)的智力,但它们展示了一系列有趣的相关功能,包括:
\begin{itemize}
  \item 基于模仿和强化来学习新的行为;
  \item 响应自然语言的命令和问题,作出恰当的行为和自然语言的回答;
  \item 自发地探索它们的世界,利用记忆来调整未来的学习和语言交互;
\end{itemize}

受游戏“我的世界”的启发,目前OpenCog正在将对虚拟狗的控制工作扩展到使用OpenCog控制游戏中的虚拟主体。这些主体最初被特别设定了各自在游戏世界中要达成的目标,为了完成这些目标,他们需要通过搬运“砖块”和使用简单的英语交流。这些任务可以是:
\begin{itemize}
\item 学习建造台阶或梯子来拿到高处的物体;
\item 学习建造掩体来躲避入侵者;
\item(?Learning to build structures resembling structures that its shown)
\item 学习建造桥梁以跨越峡谷。
\end{itemize}

当然,这一类任务中AI的显著性取决于系统能给予什么样的反馈、以及环境的复杂程度。让AI以非常独特的方式去做这些事情会相对简单,但这并非项目的主旨——目标是让系统通过涉身的经验以及少量的人类教师的反馈,学会使用泛化学习机制和泛化认知框架去完成类似任务。如果能成功,这将为今后的AGI研发提供一个非常不错的平台,就像一个可视化的和(immediately meaningful?)OpenCog的demo。

本文在写作时,项目团队的注意力集中在一些特定的任务上,包括:
\begin{itemize}
\item 观察另一主体通过建筑来到达高处的过程
\item 我们发现让虚拟主体观察另一个主体在一个不同的语境中,通过建造来到达高处的步骤,并模仿他的行为是一个不错的办法
\item 同时,如果虚拟主体需要一个特定的高处的物体,但是周围没有建造台阶所需的材料,那么尝试通过其他办法来达到高处(包括比如建造梯子或者让一个个子高的主体去帮他拿)
\item 我们发现,如果该主体需要隐藏自己有价值的物品,避免比自己更大的生物拿走它,那么他需要建造一个带有小洞的容器,这样该主体可以躲进去,但是比它大的生物无法进入。
\end{itemize}

延伸该工作到虚拟主体中,在2009-2010年间,预实验已经通过OpenCog在Nao机器人上进行过了[?]。这些涉及到将OpenCog与一个分离的控制底层感知和行动的子系统的整合。这个整合仅通过相当简单的方式进行,然而,如何进行该整合是论文[?]和[?]中讨论的话题,本文仅仅介绍到此。在此方面的工作在2013-2015年间一直在进行中,由Hong Kong Innovation in Technology Fund的一个基金赞助。

\subsection{概念背景}{Conceptual Background}
\indent CogPrime的设计研发受到一种叫“模式主义”的心灵哲学影响[?]。模式主义心灵哲学是对如何实现智能系统的统筹思考。它基于一个简单的前提,即心灵由模式组成——同时心灵是一个对它自己和世界的模式识别系统,尤其是那些关于在何种语境下、何种过程会导致何种结果的模式。然而,CogPrime受模式主义视角的指引这一点不应被过分解读。CogPrime是一个集成设计,通过若干不同的哲学、科学以及工程的想法的结合来形成,它的成功或失败并不取决于某一特定的哲学对智能的理解。

在细节上,追寻模式主义哲学会导致一系列特定的关于心灵本质的假设和结论。通过智能在复杂环境中完成复杂行为,我们发现一个认知系统的动力学会被一下两种因素影响:
\begin{itemize}
\item 自组织,通过系统动力学引起退出系统的模式来产生一个新的(???);
\item 面向目标的行为,在[?]中有严格的定义,但基本上相当于一个与环境交互的系统,该系统的行为类似于求某些可理解的简单函数的最大值。
\end{itemize}

自组织和面向目标行为应该被理解为互相协作的两个方面。举一个详细的例子,一个虚拟主体被要求通过一些砖块建造一个惊人的建筑,这是面向目标的。但是该主体如果在之前有过在自法的、无结构性的玩耍砖块的过程,那么它能更好地完成该面向目标的任务。同时这种要求它建造一个惊人的砖块结构的“创造力的推动”可能引起它去探索,以得到一些新奇的模式,而它可能将这些模式重新用于将来的非结构性的砖块游戏中。

基于以上概念,如[?]中所讨论的,我们可以假定若干主要的动力学原则,包括:
\begin{itemize}
\item 演化,作为一个主要过程,通过它,可以在很大数量的模式中挑选出一些,用于形成新模式的基础,基于一些与主体所要完成的任务相关的“适应度函数”;
\item 自生成:该过程让拥有多个模式交互的系统维持它的整体性,当系统中的一个模式开始降低其强度(?Intensity)时,一些其他的模式会增加他们的强度,以使得该遇到麻烦的模式重新开始增加它的强度;
\item 联系,给予注意力的模式,会将注意力散布一部分在其他曾经有过联系模式上。同时,根据皮尔斯的心灵定律[?],简述之,即心灵是一个互相联系的记忆网络,它的机制是,记忆中的每一个想法都是一个激活的智能体(agent),这个网络持续地与那些和它有关联的记忆发生作用 ;
\item 差别的注意力分配;评分系统。那些被评价为对达成目标更有价值的模式会被给予更多的注意力,并被鼓励参与新模式的生成;
\item 模式创造,被井架为对实现目标更有价值的的模式被变异与组合来产生新的模式。
\end{itemize}

接下来,按照[?]中列出的许多理由,我们假定智能系统中的模式网络必然引起一下大规模产生的结构
\begin{itemize}
\item 层次化网络。模式被与控制其他代表了更特化方面的模式所联系起来。
\item heterarchical网络(?)。系统维持一个关于那些曾经与其他模式相联系的模式的记忆。
\item 双网络。这两种结构被合并,通过一种机制让它们和谐共处。通过许多可能的方式,层次化地组织起一批模式,并保证层次结构中接近的模式有更多通向彼此的有意义的heterarchical连接;当然,必须有一个在层次结构中接近的模式间搜索heterarchical连接的机制。
\item 自结构。网络中模式的一部分形成了整个网络的结构的大致的图景。
\end{itemize}

CogPrime并没非直接由这些哲学规则产生;它最初通过合并人类认知心理学和计算机科学算法结构而创建,然后修改这个组合以产生一个系统,使它看起来与这些哲学规则相一致的、并且以当前的硬件基础在计算上可行,它还将包括一个大致上与人类相似的认知结构。CogPrime的成功将主要取决于这些高阶结构和机制是否能够通过CogPrime中表达和算法的系统交互中产生出来,它们将被用于在恰当的环境中控制恰当的主体。在[?]中详细讨论了这些抽象的概念如何从CogPrime的结构和算法中具体地产生出来。

\subsection{CogPrime的高阶架构}{High-Level Architecture of OpenCog}
图1描述了CogPrime的高阶架构。一个关键的潜在原则是:与多种类型的记忆相联系的多认知过程的运用,将会使得一个智能主体执行它认为在当前环境下对完成目标最有利的过程。例如,在机器人学龄前阶段的条件下,最高层的目标将会是日常的事物,如取悦老师、学习新的知识和技能、保护机器人的身体。

将这些图表与人类的认知结构图表进行对比是有趣的[?],它将概述目前所理解的人类认知结构。主要的区别在于,CogPrime的图表
