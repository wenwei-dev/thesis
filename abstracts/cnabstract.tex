\begin{abstract}

随着人工智能的不断普及,作为人工智能的一个重要目标的智能对话系统也成为现代各种电子设备及社交网络中一个不可或缺的功能,如苹果的siri,微软的小冰等,虽然目前的计算语言学研究在一定程度上提高了很多自然语言处理相关领域的应用性能,如机器翻译、文本检索、语音识别等,但对于人机交互的智能对话,其许多关键领域的技术以及表现还是与人类的需求相差甚远。

针对当前很多智能对话系统没有很好结合语境以及在对话中进行常识性推理应用不足等问题, 本文在对话系统模型和自然语言理解、生成等方面展开了深入研究,旨在实现能真正理解语言并能根据自身环境做出一定逻辑推理的智能对话系统。本文的主要工作和创新点如下:

\begin{itemize}
\item 本文在言语行为理论的指导下,将言语也看成是说话时在实施某种行为,并在设计和实现言语行为时,使用了概率逻辑来表示和实现言语行为。本文的研究发现,通过使用概率逻辑推理来实现言语行为,并使用言语行为理论的指导,能有效地模拟智能的自然语言对话。
\item 在对话的过程中,本文提出的对话系统模型使用了情感计算模型,并结合系统当前的目标、语境以及具有的知识,来选择最适合的言语行为继续对话。在该对话模型中,一旦相应的言语行为被选择用于回应谈话对象,系统会使用概率逻辑推理来确定其最好的自然语言表达方式。
\item 为了能实现上述的对话模型,本文还设计并实现了能将语言转换成逻辑表达式的自然语言理解系统。该系统使用了多层句法语义分析,将自然语言理解扩展到逻辑层次,从而能使对话系统对谈话内容做出一定的常识性推理。
\item 为了能使对话模型处理过的逻辑表达式转换成自然语言输出给谈话对象,本文设计并实现了能将逻辑表达式转换成语言的自然语言生成系统。该系统包含了根据言语行为理论设计的微观规划器,和将逻辑表达式转换成自然语言的表层生成器。
\end{itemize}

总之, 本文提出了一个全新的基于言语行为理论和概率逻辑推理的智能对话系统模型,并为该模型的实现展开了自然语言理解、自然语言生成以及推理方面的研究,设计和实现了能将英语句子转换成逻辑形式的自然语言理解系统,还设计并实现了能将逻辑形式转换成英语句子的自然语言生成系统。


\keywords{言语行为;语言推理;对话系统}

\end{abstract}
