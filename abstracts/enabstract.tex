\begin{englishabstract}
    A software program that could fluently understand, generate, and converse in natural language would have a huge variety of uses.   However, current  computational linguistics software falls far short of human performance in almost every key area required for achieving this functionality, e.g. language comprehension, generation and dialogue.   Our goal in undertaking the work reported in this thesis has been to build a conceptual and software toolset that will allow the construction of truly capable natural language processing software, with interactive dialogue as the main long-term application in mind.

We have carried out our work mainly within the OpenCog cognitive architecture, an integrated software platform designed with the goal of robust Artificial General Intelligence.   OpenCog's key feature is the representation of knowledge in a weighted, labeled hypergraph store called the Atomspace, which allows representation of knowledge in a language that is both rigorous in the sense of probabilistic logic, and relatively simple and commonsensical and amenable to automated learning mechanisms.   

On a theoretical level, the central problem we have confronted is the design of an English language dialogue system incorporating language comprehension (via mapping English sentences into logic expressions in the Atomspace format), language generation (via mapping Atomspace logic expressions into English sentences), logical reasoning on Atomspace logic expressions, and motivation-driven dialogue control.    Our practical implementation and experimentation work has focused mainly on the comprehension and generation portions of the design, which we have successfully brought from the design phase to the stage of adequate initial practical functionality.  We have also done some implementation and experimentation regarding the reasoning and dialogue control components, though our work there has been more at the research and early prototype level.

On a more technical level, regarding comprehension, generation and reasoning, we have aimed to explore the following hypotheses: 1) That it is viable to transform a wide variety of natural language expressions into logic expressions via a system of hypergraph transformations, using dependency grammar and an appropriate combination of term logic and predicate logic; 2) That it is viable to carry out a variety of simple logical inferences, using inference rules represented as hypergraph transformations and representing aspects of human commonsense reasoning, based on combining the logic expressions output by the above-described natural language comprehension framework; 3) That it is viable to transform a wide variety of logic expressions into natural language expressions, via a process comprising a system of hypergraph transformations and centered on matching against a knowledge base composed of (linguistic expression, logic expression) pairs formed via automated language comprehension; 4) That a framework incorporating items 1-3 can be utilized for natural language dialogue.

In the comprehension domain, we have demonstrated a natural language comprehension pipeline that chains together an improved version of the Carnegie-Mellon Link Parser, an improved version of the RelEx relation extractor system, and a wholly new system called RelEx2Logic that maps the output of RelEx into logic expressions in the Probabilistic Logic Networks utilized by OpenCog.    We have also demonstrated a novel natural language generation system, implemented inside the OpenCog system, which generates English sentences from a provided logic expression via a microplanning phase followed by a surface realization phase, where the surface realization process is based on matching fragment of the provided logic expression to fragments of logic expressions produced via comprehension using the above framework.   We have demonstrated the use of OpenCog's Probabilistic Logic Networks (PLN) framework used to carry out a variety of logical inference examples based on premises that are logic expressions obtained by interpreting English sentences according to the above comprehension framework.  Finally, we have shown that this integrated framework can be used for natural language question answering, a simple case of interactive dialogue.

The work reported in this thesis remains ongoing, and the long-term goal of a natural language dialogue system with full human-level functionality remains a significant way off.   However, the theoretical and practical tools created in the course of this thesis work constitute a significant step toward this long-term goal, and also possess significant value in themselves, both as tools for use in real-world software applications and as demonstrations of what kind of natural language processing it is possible to do using an OpenCog logic-based framework.

\englishkeywords{Natural Language Comprehension; Knowledge Representation; Dialogue System; Natural Language Generation}
\end{englishabstract}
