\begin{englishabstract}
  
Along with the popularization of Artificial Intelligence(AI), Dialgoue System, considering as one of the most important goal of AI, has become a tremendous required software for modern electronic devices and social networks, such as, Siri from Apple, Little Ice from Microsoft and so on.  As we known, current technology of computational linguistics research has improved a lot on various areas of Natural Language Processing, such as, Machine Translation, Text Retrieval, Speech Recognition and so forth, and has made them usable at some level. However, regarding to Intelligent Dialogue System, the related software still falls far short of human performance in almost every key area required for achieving this functionality. 

Considering the various problems that the current research on Dialogue System remained, especially the dialgoue control problem, we have carried out work mainly on designing a cognitive dialgoue model that is able to do some commonsense reasoning as well as automated learning in a weighted, labeled hypergraph store called the Atomspace, which allows representation of knowledge in a language that is both rigorous in the sense of probabilistic logic, and relatively simple and commonsensical and amenable to automated learning mechanisms.  The major works and contributions of this thesis can be summarized as follows:


\begin{itemize}

\item With the guidance of Speech Act Theory, we consider the speech as an action that the speaker wants to take, thus the dialogue control can be treated in a way that the agent uses for other behavioral actions. In particular, we have defined different control machanisms for different dialogue acts, which are integrated with a probabilistic logic model called PLN(Probabilisitic Logic Networks) and a motivation-driven emotion computational model called Psi, to design a more robust and intelligent dialogue model. We have also implemented some dialogue control machanisms for the question answering dialogue act, based on the cognitive dialogue model we designed.

\item In order to understand the conversation and gain knowledge from natural language corpuses, we have designed and implemented a Natural Language Understanding pipeline, which can transfer natural language sentences into the hypergraph represented logic relationships and stores in our weighted, labeled hypergraph knowledge base Atomspace. With this achievement, we have also implemented a logic reasoning system with the input and output of natural language sentences.

\item In order to generate what the system wants to express, we have designed and implemented a Natural Language Generation pipeline, which can do the other way around, that is, to transfer hypergraph represented logic relationships into natural language sentences.

\end{itemize}

The work reported in this thesis remains ongoing, and the long-term goal of a natural language dialogue system with full human-level functionality remains a significant way off.   However, the theoretical and practical tools created in the course of this thesis work constitute a significant step toward this long-term goal, and also possess significant value in themselves, both as tools for use in real-world software applications and as demonstrations of what kind of natural language processing it is possible to do using our logic-based framework.

\englishkeywords{Knowledge Representation; Language Reasoning; Dialogue System; Natural Language Comprehension; Natural Language Generation}
\end{englishabstract}
